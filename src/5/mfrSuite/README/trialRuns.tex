\documentclass[10pt,a4paper]{article}
\usepackage[utf8]{inputenc}
\usepackage{amsmath}
\usepackage{amsfonts}
\usepackage{amssymb}

\title{Trial Run Questionnaire and Guide}
\usepackage[margin=1in]{geometry}
\author{Joseph Huang}
\date{}
\begin{document}
\maketitle
\section{Questionnaire}
This exercise consists of two parts. First, we ask you to run an existing model in the literature. Second, we ask you to run a model of your own. \textbf{If you ever encounter a difficulty or hurdle such that it takes too long for you to figure out, please stop and email me. This software is meant to be easy, concise, and self-explanatory. Therefore if it takes too much time to figure out what to do, it is probably my fault!}

\subsection{Solve an existing model}

In this model, can you solve a model with parameters: $a_h = -\infty, \gamma_e = \gamma_h = 3, \psi_e = \psi_h = 3, \phi = 5, \underline{\chi} = 0.4$. This parameterization mimics He and Krushnamurthy (2013). \\


Can you please take note of the following:
\begin{itemize}
\item Any unclear, confusing parts in the documentation?
\item Were you stuck anywhere, such that you had to do a lot of Googling to figure out?
\item Any bugs?
\item Amount of time, in total, needed to figure out how to complete this exercise?
\item Amount of time taken for the program to run?
\item The outcome of the program: (1) Was it clear to you what happened? (2) What actually happened?
\item Suggestions from programming and user-experience perspectives
\item Anything else?
\end{itemize}

\subsection{Your own model}

In this exercise, please pretend that you're using this software for your research purposes and that you're testing out a model of your interest. Can you please, in addition to the bullet points above, take note of the following:
\begin{itemize}
\item The models that you tried to solve
\item What did you do in order to solve your model of interest?
\item Outcomes of your attempts (did it converge or did it get stuck?)
\item Were you able to tell what happened when your attempt failed to generate what you wanted?
\item Were you able to find a way to make the program converge, when your first attempt failed? If so, how did you figure out what to do, and what did you do exactly?
\end{itemize}
%
%Can you solve a set of two models:
%
%\begin{itemize}
%\item Model 1: $a_h = -\infty, \gamma_e = \gamma_h = 3, \psi_e = \psi_h = 3, \phi = 5, \underline{\chi} = 0.4$.
%\item Model 2: same as above, but change $\gamma_h$ to $9$.
%\end{itemize}
%
%while leaving other parameters as default.\\
%
%After solving the model, can you find
%\begin{itemize}
%\item Stationary density
%\item Ergodic mean and standard deviation of wealth share and experts' equity retention
%\item Shock elasticities for 50 periods by fixing the starting point at $w \in \{0.1, 0.5, 0.9 \}$
%\end{itemize}
%
%\subsection{Brunnermeier and Sannikov 2014}
%
%Model parameters: $a_h = 0.5a_e, a_e = 0.14, \gamma_e = \gamma_h = 3, \psi_e=\psi_h =3, \phi = 5, \underline{\chi} = 1$ while leaving other parameters as default.
%
%\subsection{Gârleanu and Panageas 2015}
%
%Model parameters: $a_h = a_e, \psi_e =1, \psi_h = 8, \gamma_h = 1, \gamma_e = 8$ while leaving other parameters as default.
\end{document}